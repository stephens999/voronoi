\documentclass[10pt,titlepage,times,letterpaper]{article}
\usepackage{enumerate}
\usepackage{epsf}                
\usepackage{natbib}
\usepackage{graphicx}
\newtheorem{algorithm}{Algorithm}[section]
\usepackage{amsmath}
\usepackage{color}
%define colors
\definecolor{mediumblue}{rgb}{0.0509,0.35,0.568}
\definecolor{blue}{rgb}{0.0109,0.15,0.468}
\definecolor{lightblue}{rgb}{0.81176,0.92549,0.9333}
\definecolor{yellow}{rgb}{0.961,0.972,0.047}
\definecolor{red}{rgb}{0.9,0.1,0.1}
\definecolor{orange}{rgb}{1.0,0.4,0.0}
\definecolor{darkblue}{rgb}{0.109,0.35,0.538}


\def\SCAT{{\tt SCAT} }
\def\VORONOI{{\tt VORONOI} }
\def\Di{{\cal D}}

\def \Pr{{\rm Pr}}
\def \E{{\rm E}}
\def \Var{{\rm Var}}
\def \rhobar{{\bar{\rho}}}

\pagenumbering{arabic}
\begin{document}

\title{Documentation for {\tt VORONOI}, version 2.0}

\author{
Software code by Matthew Stephens\footnote{email: {\tt mstephens@uchicago.edu}} \\
WWW: {\tt http://github.com/stephens999/voronoi}\\
\\
}

\date{Revised by Mary Kuhner\footnote{email:  {\tt mkkuhner@uw.edu}} July 29, 2021}


\maketitle

\tableofcontents
\vfil\eject
\section{Introduction}
The \VORONOI program was written by Matthew Stephens and members of his lab.  Mary Kuhner
and Jon Yamato in the Wasser Lab, Center for Conservation Biology
have taken over maintenance and expansion of the program, and wrote this documentation.
Some of the statements in the documentation are our best deduction from the source code
and published paper.  If you find any errors please contact us at {\tt mkkuhner@uw.edu}.
Locations to obtain the most recent version of the software, and instructions for citing
it, are in section \ref{software}.

\VORONOI is a postprocessor for output from the \SCAT program (Wasser et al. 2004, see
section \ref{software}).
It requires \SCAT to have been run in the mode where
it assigns the locations of individuals based on their genetic data.  \VORONOI attempts
to refine the estimates by assuming that the location-unknown individuals are likely to
be clustered relative to the entire reference set of individuals.  It was designed to
tighten estimates of the locations of elephant tusks from the same ivory shipment, which
may represent a small region within the elephant species range.  The algorithm is
described in Wasser et al. (2007).

\VORONOI is appropriate when there is reason to think that the unknown samples are
geographically clustered relative to the reference samples, for example
because they were taken in the same shipment, found in the same marketplace, or
obtained from the same collecting expedition.  It is likely not helpful in other
situations.

{\bf Two bugs which impacted correctness of the \VORONOI search were found in version 1.
Furthermore, a questionable decision in the math was corrected.
We strongly recommend rerunning any analyses done with version 1.}  The bugs and
questionable math tended to tighten clustering of the observations; in a small
simulation (not published) we found that correcting them improved accuracy of
assigning locations to location-known individuals.  For details, see section \ref{changes}.


\section{Compiling \VORONOI}

Executables are provided for Linux only; for other systems you will
need to compile it yourself.  The provided executable is known to run
on Ubuntu 18.04.5 but has not been tested elsewhere.  The executable
is on GitHub in the main directory of the repository, with a name like
{\tt voronoi.linux.X.tar.gz}.  To unpack it:

{\tt gunzip voronoi.linux.X.tar.gz} \\
{\tt tar -xvf voronoi.linux.X.tar} \\

This will create a new directory called voronoi.linux.X.  Change to this
directory before running the program, or move the program elsewhere.
(It has very specific needs for input files to be found in its directory,
as discussed below.)

\VORONOI is a C++ program and you will need the
GNU c++ compiler (g++) or equivalent to compile it.  (If you use a different
compiler you will need to modify the Makefile.)

Once you have checked out \VORONOI from GitHub, go to the {\tt src} directory
and type ``make" to compile it.  The executable will be named \VORONOI.

As distributed, \VORONOI will be compiled in optimized format, which is
fastest to run but does not work with a debugger.  If you need a debug
version, edit the Makefile, commenting out the line that reads:

{\tt CFLAGS = -O3 \$(INCLUDE)}

and commenting in the line that reads:

{\tt \#CFLAGS = -g \$(INCLUDE)}

Once this is done, type "make clean" followed by ``make" to recompile.  
This will produce a version with debugging information, and will also
enable internal correctness checks, which will generally
halt the program with a screen error message if they detect a problem.  
This version is too slow for routine use but can be very helpful for debugging.
Please report any bugs you uncover to {\tt mkkuhner@uw.edu}.

To go back to the optimized version, revert to the original Makefile
and do ``make clean" and ``make" again.

For different OS variants you may need to comment in one of 
the LDFLAGS lines in the Makefile or make further changes.  The 
distributed Makefile is known to work as-is on Ubuntu 18.04 and 
20.04.1, and CentOS 7.7.1908.  


\section{Running \VORONOI}

\VORONOI is a command-line program and can be run with the following command:

./\VORONOI [options] controlfile PREFIX

The format of {\tt controlfile} is described in the input data section.  It is essentially a list
of which individuals (by number) are to be analyzed.  The {\tt PREFIX} is an arbitrary string (without
spaces or hyphens) that will be attached to all of the output files from this run so that they can
easily be recognized.

\VORONOI also requires appropriately pre-processed \SCAT output files to be present in the
directory where it is invoked, as described below.

\VORONOI is fairly fast; in our experience, for a case where \SCAT runs took a day or 
more, the corresponding \VORONOI run will take a few hours.  Runtime is proportional to
the number of individuals being studied.

\section{Input data}

\subsection{Pre-processed \SCAT data}

\VORONOI currently has a highly specific input format.  It requires the output of exactly
9 \SCAT runs on the same input data set, but with different random numbers.  Furthermore,
each \SCAT run must be set to produce eactly 200 samples (of which the first 100 will be
discarded as burnin).  We recommend setting up 9 directories and running \SCAT in each.

Once the 9 \SCAT runs are finished, their output must be consolidated into the directory
where \VORONOI will be run.
\SCAT writes one location file per individual, named with that individual's ID. 
To convert to \VORONOI input, the 9 groups of \SCAT
results are assigned tags ``r" through ``z" (lower case).  Each individual is 
given a number; numbers of less than 3 digits are padded on the left with zeros.  Thus, the
third individual could be coded as 003.  (The numbers are arbitrary and need not 
be consecutive.)  Each \SCAT output file must be copied into the working
directory with a new name consisting of that individual's number
and the letter for that group:  in the case of the third individual, these files would
be named 003r, 003s ... 003z.  File 003r is the \SCAT results for the third individual from
the first \SCAT run.  (We strongly recommend copying, not renaming, as it reduces
the chance you will become confused about which number is which individual.)

\subsection{Control file}

You will also need to prepare a control file which informs \VORONOI of the
individuals to be analyzed.  This is a plain text file containing exactly one line of
text.  The first number in the file gives the count of individuals to be run.
After this, on the same line, are the numbers of the individuals to be run (using the
numbering scheme used to name the files, except that padding with zeroes is not
required).  So the following control file:

{\tt 3 1 2 4}

tells \VORONOI to run 3 individuals, numbers 1, 2 and 4.  
With this input file, \VORONOI will read \SCAT data files 001r...001z, 002r...002z, and 004r...004z.
It will ignore all other files that may be present in the directory, allowing you to run different 
combinations of individuals without having to add or remove files.  The results in the \VORONOI
output files will be in the same order as the numbers listed in the control file; we strongly
recommend against listing the individuals out of numeric order due to the high probability
of getting confused in later analysis of the results.

\section{Boundary specification}

\VORONOI needs to know the boundaries within which individuals could possibly be found. {\it This boundary
specification must match that used in the \SCAT runs.}  As a result, if you have old \SCAT runs you
will not be able to use grid file boundary specification, as this was only added to \SCAT in version 3.0.
The boundary can be specified in three ways, with the most useful listed first.

\subsection{Grid file ({\tt -g})}

Inform the program of the use of a grid file with the -g option:

\medskip
{\noindent
{\tt -g gridfilename}
}
\medskip

Internally, \VORONOI represents the area under study as a grid with squares 1 degree in latitude
and longitude, and aligned on the latitude/longitude lines.  The grid file allows the user to
specify exactly which grid squares are allowable locations for sampled individuals.  A major
advantage of the grid file is that it allows the species range to be discontinuous:  for example, 
an organism found on several islands in a chain, but not in the ocean between them. 

The grid file should contain the coordinates of each grid square that can contain individuals,
one per line, listed as integer values of latitude and longitude. 
The latitude and longitude given will be interpreted as the southwest
corner of the square, so the northeast corner will have latitude and longitude 
one greater than given.  For example:

\medskip

{\noindent
{\tt
-11 3 \\
}}

specifies that the species can be found in the square whose southwest corner is at 11 S and 3 E,
and whose northeast corner is at 10 S and 4 E.  It does not matter in what order the squares are listed.

The program will do its inference, internally, on a square grid large enough to hold all of the
listed squares plus a minimum 7 degree border around the edges.   The border is considered 
outside the species range and is included to prevent the inference from being perturbed by
edge effects.

The code for grid file handling does not handle the poles or the line opposite the Prime 
Meridian 
correctly.  We recommend modifying the latitudes and longitudes to move them away from this 
area (don't forget to move them back before interpreting the results).  Also, the 1 degree 
grid ``squares" will become far from square in areas close to the poles and this may affect 
accuracy, although 
we believe this will be a bigger issue for \SCAT than \VORONOI.  All of our practical 
experience is with African elephants, which occur fairly far from the poles.

\subsection{Polygon boundary file ({\tt -b})}

This option allows you to specify the boundary of the species range as an arbitrary 
polygon.  The polygon is specified by entering the latitude and longitude of
each consecutive vertex into a ``boundary'' file (see below), and using
the {\tt -B} option to tell the software the name of this file.
Boundary file use is indicated by the -B option:

{\tt -Bboundaryfilename}

Note that (for backwards compatibility) there is no space between the B and the file name.
The boundary file should contain one row for each vertex in the
polygon, with two numbers on each row (decimal latitude and longitude,
with N and E being positive, S and W being negative) separated by a
space. The vertices should be entered in order (in either direction
around the polygon), with coordinates of the last vertex being an {\it
exact} repeat of the coordinates of the first vertex.

For example, to specify a square region, from $5^\circ N$ to $3^\circ S$ and
$2^\circ W$ to $1^\circ E$ the file could be
\begin{verbatim}
5 -2
5  1
-3 1
-3 -2
5 -2
\end{verbatim}

This code was ported from \SCAT.  We prefer use of the grid file as it does not
require that the species range be a single contiguous polygon.  While the boundary
file approach may appear capable of more precision, \VORONOI operates on a grid of
one degree squares internally and cannot make use of the additional precision.

The routine that tests whether a point is inside or outside
the specified polygon is based on the 2-D algorithm 
described by Dan Sunday ({\tt http://www.softsurfer.com}). 
It will not work for regions that span
$180^\circ E$ (the line opposite the prime meridian) unless you enter
the Longitudes as having the same sign, eg by using $190$ and $170$
instead of $-10$ and $170$. It also will not work for regions
including the North or South Pole.   We recommend recoding such
data and reversing the recoding before interpreting the results.  It
is possible that no usable recoding exists for a circumpolar or very widespread species.

\subsection{Inbuilt boundaries ({\tt -D, -d})}

\VORONOI was written to analyze African elephant data, and contains hard-coded
boundaries for savannah and forest African elephants.  These boundaries are 
imperfect--the savannah boundary contains large areas of ocean--and we do not
recommend their use for anything other than deliberate replication of previous results.

The savannah hard-coded boundaries can be accessed with {\tt -d}, and the 
forest boundaries with {\tt -D}.  No auxillary input file is needed.  Consult
the output {\it PREFIX\_mapinfo} file to see details of the grid used.

The same options exist in \SCAT, with the same caveats; if you use them in one
program you must use them in the other as well.

\section{Other useful options}

The following options are specified on the \VORONOI command line.

\subsection{Setting the random number seed  ({\tt -S})}

The {\tt -S } option allows you to specify the random number seed, which should be a
positive integer.  If no seed is set, one will be derived from the system clock.
Setting the seed is useful to allow you to replicate a run:  for example, debugging is
much easier if you are running the same execution path each time.
It is also useful when multiple copies of the program are being launched by a script,
as if they are launched too quickly they may get the same clock time and thus the same seed.

\subsection{Alternative output format  ({\tt -v})}

The verbose option {\tt -v} will cause \VORONOI to write an additional output
file ({\tt printprobs}) presenting the same results as the {\tt indprobs} file, but differently formatted (longer
but easier to parse).  For details see the output section.

\section{Experimental options}

\subsection{Likelihood cutoff ({\tt -C})}

\VORONOI normally considers a clustering solution valid if it explains at least
one \SCAT observation for each individual.  This can lead to over-clustering:  if there
is a tight cluster of many individuals, and the remaining individuals have very weak but
non-zero support for falling in the cluster, \VORONOI is liable to put all of the individuals
there.

This behavior can be changed with the cutoff {\tt -C} option.  Follow the option with
a decimal number between zero and one inclusive.  Only solutions which explain more
than this proportion of \SCAT observations for each individual will be considered valid.  
A cutoff
of zero emulates the previous behavior of \VORONOI and is equivalent to having no
cutoff at all.  A cutoff of one means that all \SCAT results for every individual must
be included in the \VORONOI solution, which in practice means that \VORONOI will nearly
replicate the \SCAT solutions.  Very limited
work has been done to find whether setting a non-zero cutoff improves \VORONOI results,
but it is clear that appropriate cutoffs are likely to be quite small:  in our test
data, cutoffs greater than 0.2 produce results which are essentially the same as 
the underlying \SCAT results and therefore add nothing to them, and the best results 
seen are for much lower but non-zero cutoffs such as 0.02.

\section{Outputs}

\VORONOI writes several files. Each filename will begin with the PREFIX
specified in the run command, so that results from the same run can be easily
recognized.


\subsection{Map info file}

This file is named {\tt PREFIX\_mapinfo}, and contains information for interpreting the
\VORONOI map grid:  this is essential for correct interpretation of the results.
\VORONOI works in 1 degree latitude and longitude squares, and refers to each
square by the decimal latitude and longitude of its south-western corner.  Thus,
the grid square at -14, 21 (14 degrees South, 21 degrees East) extends from -14,21
to -13,22.  It also gives the number of squares in the grid, which can be helpful
in post-processing the results.  Currently the grid must be square, though we plan
to remove this restriction in a later release.

\subsection{Indprobs file}

This file is named {\tt PREFIX\_indprobs}, and contains the posterior probability
for each grid location, for each sample in turn. If the grid is, for example, 67 x 67,
the first 67 lines will correspond to the first sample listed in the input file,
the second 67 lines will correspond to the second sample, and so forth.  Each entry
gives the posterior probability that the given individual is found in the given
grid square.

Note that the order of entries is the order in the input file; if the input file is
out of numerical order, the output will be too (this is not recommended!)  You
will want to keep very careful track of which individual corresponds to each entry.
A future project is having \VORONOI read \SCAT input directly and retain the sample
IDs, to avoid problems of this kind.

Note that the first grid square written out is the LOWER LEFT grid square, even though
it will fall in the upper left part of the printout.  If your results seem upside
down, it's likely because you are interepreting the results as if 0,0 was the square
in the UPPER LEFT, which is visually suggested by the way the output is printed.
If you find this confusing (we do) we recommend turning on the {\tt -v} option,
which writes the output in a different format.

The maximum entry for a sample can be taken as a point estimate of its location, but
we recommend looking at the whole grid.  A useful way to do so is to plot it as a heatmap
onto a world map of the area in question.  \VORONOI placements can consist of several
well separated plausible regions, and a single point estimate does not represent
such data well.

\subsection{Printprobs file}

This file is only written if the verbose {\tt -v} option is given, and is named
{\tt PREFIX\_printprobs}.  It contains exactly the same information as 
{\tt PREFIX\_indprobs}, but presented in a different way.  Roughly, indprobs is good for
spreadsheet users and R programmers, printprobs is good for other users.

In the printprobs file each individual in turn is represented by a number of lines
equalling the number of squares in the grid.  Each line gives the decimal latitude and 
longitude of a grid square and the posterior probability for that individual to be
in that grid square.  Thus, it is not necessary to convert between grid square
index and latitude/longitude coordinates in order to interpret this file.

\subsection{Regions file}

This file is named {\tt PREFIX\_regions}.  It contains one grid per non-burnin
iteration of \VORONOI.  These grids are dimensioned in the same way as the ones in
{\tt PREFIX\_indprobs} but contain only 1's and 0's.  A square is 1 if it is both within
the species range and within the current estimate of R, the region inferred to contain all
the samples.  Thus, one could plot these to see how 
R evolved over the course of the run.

\subsection{Overall probabilities file}

This file is simply named {\it PREFIX}.  It contains one grid, dimensioned the same
way as the ones in {\tt PREFIX\_indprobs}, but containing the posterior probability, for
each grid square, that it was live (within both the region (called R in the paper)
in which all samples were
inferred to lie, and the species range), taken as an average over all iterations.  
Surprisingly, this includes burn-in iterations.  The contents of this file can be taken
as an estimate of R.  

\subsection{Trace file}

This file is named {\tt PREFIX\_trace}.  It contains two columns:  the first is iteration
number, and the second is the value, at that iteration, of the metaparameter {\tt p}.
This parameter estimates what proportion of Voronoi polygons will fall within the
species range.  The file can be examined to see if the search over this parameter is
mixing well, and what the range of inferred values of {\tt p} is.  A useful tool
for this purpose is the TRACER program (see \ref{software}).  All iterations are
listed, including burnin.  If an estimate of {\tt p} is desired, the mean or median of
the non-burnin iterations are reasonable estimates.

If, for a given iteration, the posterior probability for that iteration is calculated
as less than or equal to zero, or greater than or equal to one, no entry is made in the trace 
file for that iteration.
On our data this is less than 1\% of iterations.  Such iterations
are always rejected, so if a full trace is desired, missing entries can be replaced by
the entry immediately before them.  This is likely to be a bug but seems fairly
harmless.

\section{Limitations of the software} \label{limitations}

In our experience, when the majority of individuals in the sample come from a restricted
location but there are a few sporadic individuals from elsewhere, \VORONOI tends to put
the sporadic individuals with the others if there is any support for such a placement at
all.  The experimental ``cutoff" option may possibly help in this situation.   Sporadics
can be detected by comparing the raw \SCAT results with the \VORONOI results:  an individual
which is placed in the \VORONOI main cluster despite very low \SCAT support for this placement
should be regarded skeptically.

\VORONOI must use the same species boundaries as the \SCAT runs on which it relies, and
will not necessarily diagnose violations of this rule.  In general it does very little error
checking.  Omitting one of its \SCAT input files, for example, will likely cause it to
crash without explanation.

We do not believe \VORONOI is helpful unless there is some reason to believe that the
unknown samples will be clustered relative to the reference ones:  the raw \SCAT results
are likely preferable if there is no reason to expect clustering.

\VORONOI is a \SCAT postprocessor and relies on \SCAT having performed well.
Consult the \SCAT documentation for advice on diagnosing and fixing issues with \SCAT.
While it might be tempting to hope that \VORONOI will improve bad \SCAT results, we
see no reason to expect that it will.  Be particularly watchful for hybrids, which are
highly disruptive to \SCAT.

\section{Changes between 1.0 and 2.0} \label{changes}

The original \VORONOI code did not have a version number but will be called version 1.0.

\subsection{Corrections}

\subsubsection{Coherency bugs}  
Two serious bugs present in 1.0 were corrected in 2.0.  Both 
caused incoherency of the program state on a particular accepted or rejected MCMC
step.  Specifically, the program maintained both a list of which polygons were
active, and a count of how many grid squares were in active polygons.
The code contained two different routines in which the count of grid squares was not 
correctly updated and would
disagree with the information in the list of polygons.  We believe that these bugs 
caused a bias towards tighter clustering (fewer active polygons) as well as
adding noise to the analysis.  We recommend re-running any results obtained with 1.0.

\subsubsection{ Inference of $p$} 
The mathematical model was changed between 1.0 and 2.0.   
The 1.0 behavior may have been
intended but we believe it to be incorrect.   Specifically, the metaparameter
$p$ gives the probability that a particular Voronoi polygon will be included within
the putative region R from which the samples were drawn.  In version 1.0, polygons which
fell totally outside the species range, and could thus never be included in R,
were scored when proposing changes to $p$.  This caused a downwards bias in $p$,
especially if the species range was much smaller than the \VORONOI grid, and this bias
led to over-clustering (a too-small R).  In 2.0, polygons outside the species range
are not considered in changing $p$.  In elephant data we have found this to improve
accuracy, evaluated by estimating the location of individuals for whom the sampling
location is known. 

\subsubsection{ Disagreement between \SCAT and \VORONOI}  
When \SCAT and \VORONOI are 
run with either the hardcoded range boundaries, or range
boundaries set as a polygon bounding box, they can disagree over the legality of an 
individual's placement:  \SCAT asks whether the individual's exact placement is in-bounds,
whereas \VORONOI asks whether the center of the 1 degree grid square in which that
placement falls is in-bounds.  Version 1.0 ignored this issue:  we are not sure what the
effect of these out-of-bounds results was on the resulting estimates.
In version 2.0, \SCAT results which fall slightly outside the
\VORONOI boundary are moved into an adjacent in-boundary square; if no such square is
available the program will terminate.  (In our experience, if the program terminates for
this reason you have probably used different boundaries in \SCAT and \VORONOI and need
to correct that.)   This issue does not arise with use of grid square
range boundaries as the identical rule for legality is then used in both programs.

It should also be noted that since the species boundary must match in \SCAT and \VORONOI,
and \VORONOI 1.0 did not accomodate boundary files, all runs made using a boundary file
in \SCAT and then analyzing the results with \VORONOI 1.0 were incorrect.  The
likely behavior is that \SCAT results falling outside the \VORONOI boundaries were
silently discarded, reducing statistical power.  (The program would crash
if all of the \SCAT results for a given individual were outside the
\VORONOI boundaries, but even one in-range \SCAT result would let the run proceed.)

\subsection{Additions}

This manual was added to supplement the original ``readme" file.

Version 1.0 could not analyze more than 999 samples due to requiring a three-digit
ID code.  Version 2.0 can analyze an arbitrary number of samples.  Zero-padding 
to exactly three digits is still required for 1 or 2 digit ID codes.

Version 1.0 was specialized to work on savannah or forest elephants and had hard-coded
species boundaries appropriate to these taxa.  Version 2.0 can accept boundary information
(as either a set of polygon vertices or a set of grid square coordinates) for
any desired species range as long as it does not include either pole or cross the line
opposite the Prime Meridian.
{\bf It is essential to use the same species boundary for the \SCAT runs and the \VORONOI run.}
Use the {\tt -b} option for a set of polygon vertices or the {\tt -g} option for
a set of grid square coordinates.

Version 1.0 always used a 67 x 67 grid of one-degree squares.  Version 2.0 uses this
grid if the built-in defaults or a polygon boundary file are used, but if a grid file is used,
\VORONOI will instead determine the size of its grid by taking the smallest square region 
that contains all of the specified grid squares and then adding a 7 square boundary on all 
sides.  
{\bf If you have written code, scripts, or spreadsheets that process \VORONOI output, they
probably expect a 67 x 67 grid and will need to be modified accordingly.}  We consider
the cost in backwards compatibility to be justified by the ability to analyze areas 
smaller or larger than the African elephant range in a straightforward fashion.

Limited error checking of input files and data has been added.

Two new command line options were added:  verbose output ({\tt -v}) and the experimental cutoff 
option ({\tt -C}). 

Two new output files were added:  the {\tt PREFIX\_mapinfo} file gives information on the
size and geographic location of the \VORONOI grid, and the optional {\tt PREFIX\_printprobs}
file repeats the information from {\tt PREFIX\_indprobs} in a more verbose format.

\subsection{Removals}

An option {\tt -t} which had been orphaned in the code, and did nothing, was removed.

\section{Software and publications}\label{software}

When using this program, {\it please specify the version of the software you used,} and
cite the following publications:

\begin{itemize}
\item \VORONOI \\
Source code: {\tt http://github.com/stephens999/voronoi} \\
Publication:

Wasser SK, Mailand C, Booth R, Mutayoba B, Kisamo E, Clark B, Stephens M (2007)
Using DNA to track the origin of the largest ivory seizure since the 1989 trade ban.
PNAS 104: 4228-4233.

\item \SCAT \\ 
Source code:  {\tt http://github.com/stephens999/scat} \\
Publication:

Wasser SK, Shedlock AM, Comstock K, Ostrander EA, Mutayoba B, Stephens M (2004)
Assigning African elephant DNA to geographic region of origin:  applications
to the ivory trade.  PNAS 101: 14847-14852.
\end{itemize}


The following third-party program may be useful with \SCAT and \VORONOI.  It
plots traces of an MCMC run over time and can help diagnose search problems.

\begin{itemize}
\item TRACER \\ 
Source code: {\tt https://beast.community/tracer } \\
Publication:

Rambaut A, Drummond AJ, Xie D, Baele G, Suchard MA (2018)  Posterior summarization in Bayesian
phylogenetics using Tracer 1.7.  Syst Biol.  syy032.  doi:10.1093/sysbio/syy032.
\end{itemize}
\end{document}
